\documentclass[12pt]{article}
    \usepackage{times}
    \usepackage[latin1]{inputenc}
    \usepackage[frenchb]{babel}
    \usepackage{graphicx}
    \usepackage[numbers]{natbib}
    \usepackage{amsfonts}
    \usepackage{amsmath}

    \numberwithin{equation}{section}
    \numberwithin{figure}{section}
    \numberwithin{table}{section}

    \usepackage[nooneline]{subfigure}
    \usepackage{bm}
%    \usepackage{prelim2e}
    \usepackage{booktabs}
    \usepackage{color}
    \usepackage{rotating}
    \usepackage{afterpage}
%    \usepackage[draft]{fixme}
    \usepackage{refstyle}
        \newref{equ}{refcmd=(\ref{#1}),name=�quation~,Name=�quation~,names=�quations~,Names=�quation~,lsttxt={\ et\ },rngtxt={\ �\ }}
        \newref{fig}{name=figure~,names=figures~,Name=Figure~,Names=Figures~,lsttxt={\ et\ },rngtxt={\ �\ }}
        \newref{subfig}{name=figure~,names=figures~,Name=Figure~,Names=Figures~,lsttxt={\ et\ },rngtxt={\ �\ }}
        \newref{chap}{name=chapitre~,names=chapitres~,Name=Chapitre~,Names=Chapitres~,lsttxt={\ et\ },rngtxt={\ �\ }}
        \newref{sect}{name=section~,names=sections~,Name=Section~,Names=Sections~,lsttxt={\ et\ },rngtxt={\ �\ }}
        \newref{ssect}{name=section~,names=sections~,Name=Section~,Names=Sections~,lsttxt={\ et\ },rngtxt={\ �\ }}
        \newref{sssect}{name=section~,names=sections~,Name=Section~,Names=Sections~,lsttxt={\ et\ },rngtxt={\ �\ }}
        \newref{tab}{name=tableau~,names=tableaux~,Name=Tableau~,Names=Tableaux~,lsttxt={\ et\ },rngtxt={\ �\ }}
        \newref{step}{name=�tape~,names=�tapes~,Name=�tape~,Names=�tapes~,lsttxt={\ et\ },rngtxt={\ �\ }}
        \newref{annex}{name=annexe~,names=annexes~,Name=Annexe~,Names=Annexes~,lsttxt={\ et\ },rngtxt={\ �\ }}
    \usepackage{fancyhdr}
    \usepackage{wrapfig}
    \usepackage{url}

    \setlength{\textwidth}{175mm}
    \setlength{\textheight}{210mm}
    \setlength{\headsep}{2mm}
    \setlength{\topmargin}{-3mm}
    \setlength{\unitlength}{1mm}
    \setlength{\itemsep}{0mm}
    \setlength{\columnsep}{0.3125in}
    \setlength{\headheight}{10mm}
    \setlength{\parindent}{1pc}
    \setlength{\oddsidemargin}{-.304in}
    \setlength{\evensidemargin}{-.304in}
    \setlength{\abovedisplayskip}{-2mm}
    \setlength{\belowdisplayskip}{-2mm}
    \setlength{\abovedisplayshortskip}{-2mm}
    \setlength{\belowdisplayshortskip}{-2mm}
    \setlength{\topsep}{0mm}
    \addtolength{\headsep}{10mm}

%    \renewcommand{\subfigcapskip}{3pt}
%    \renewcommand{\subfigtopskip}{2pt}
%    \renewcommand{\subfigbottomskip}{3pt}
%    \setlength{\textfloatsep}{6pt plus 3pt minus 3pt}
%    \setlength{\floatsep}{6pt plus 3pt minus 3pt}
%    \setlength{\abovecaptionskip}{5pt}


    % Allows to have list item with less space between each element of the list
    % Use \noitemsep or \doitemsep before
    \let\origEnumerate =\enumerate
    \let\origItemize =\itemize
    \let\origDescription =\description
    \def\Nospacing{\itemsep=0pt\topsep=3pt\partopsep=0pt%
    \parskip=0pt\parsep=0pt}
    \def\noitemsep{% Redefine the environments in terms of the original values
        \renewenvironment{itemize}{\origItemize\Nospacing}{\endlist}
        \renewenvironment{enumerate}{\origEnumerate\Nospacing}{\endlist}
        \renewenvironment{description}{\origDescription\Nospacing}{\endlist}
    }
    \def\doitemsep{% Redefine the environments to the original values
        \renewenvironment{itemize}{\origItemize}{\endlist}
        \renewenvironment{enumerate}{\origEnumerate}{\endlist}
        \renewenvironment{description}{\origDescription}{\endlist}
    }

    % Allows to ignore continued figure in the TOC
    \makeatletter
    \def\afterfi#1\fi{\fi#1}
    \let\ORIGaddtocontents\addtocontents
    \newcommand*\dontaddtolof[2]{\edef\temp{#1}%
    \ifx\temp\ext@figure\else\afterfi\ORIGaddtocontents{#1}{#2}\fi}
    \makeatletter
    \newcommand*\ignorelof{\let\addtocontents\dontaddtolof}
    \newcommand*\obeylof{\let\addtocontents\ORIGaddtocontents}

    \bibliographystyle{plainnat-fr}

%    \linespread{1.6}

    \pagestyle{fancy}

    \lhead{\small{DI, Universit� de Sherbrooke}}
    \rhead{\small{g�n�ration proc�durale de b�timents}}

\begin{document}

\title{G�n�ration proc�durale de b�timents}

\author{Alexandre Brochu}
\date{24 / 04 / 2015}

\maketitle

\thispagestyle{empty}

\vspace*{-5mm}

\begin{center}
\small{
    D�partement d'informatique \\
    Universit� de Sherbrooke \\
    Sherbrooke (Qc), Canada, J1K 2R1 \\
    Alexandre.Brochu@usherbrooke.ca
}
\end{center}

\vspace*{5mm}

\begin{center}
\textbf{- Rapport de recherche no 1 -}
\end{center}

\vspace*{5mm}

\begin{abstract}
    le travail reli� � la cr�ation de mod�les 3D de villes pour cr�er un r�alisme dans un environnement urbain dans les films ainsi que dans les jeux vid�o devient de plus en plus complexe. On recherche des fa�ons d'automatiser ce travail � l'aide d'algorithmes qui se basent sur la g�n�ration proc�durale. Ici, on parle d'un de ces algorithmes un peu plus en d�tail.
\end{abstract}

%-------------------------------------------------------------------------

\renewcommand{\refname}{Bibliographie}
\renewcommand{\listfigurename}{Liste des figures}

\eject

    \tableofcontents

\eject

    \addcontentsline{toc}{section}{\listfigurename}
    \listoffigures

\eject

%-------------------------------------------------------------------------

%%%%%%%%%%%%%%%%%%%%%%%%%%%%%%%%%%%%%%%%%%%%%%%%%%%%%%%%%%%%
\section{Mise en contexte}
\label{sect:miseEnCtx}
%%%%%%%%%%%%%%%%%%%%%%%%%%%%%%%%%%%%%%%%%%%%%%%%%%%%%%%%%%%%

%You can use~\citep{deriche95, tschumperle02, weickert97} or~\citet{deriche95,
%tschumperle02, weickert97} or~\citeauthor{deriche95, tschumperle02, weickert97}
%or~\citeyear{deriche95, tschumperle02, weickert97}.

%Example of equation
%\begin{equation}\label{equ:HeatConserv}
%    \nabla \cdot (\kappa \nabla u) = f.
%\end{equation}

%Reference to \equref{HeatConserv}.

%Reference to \sectref{one}.
%%%%%%%%%%%%%%%%%%%%%%%%%%%%%%%%%%%%%%%%%%%%%%%%%%%%%%%
On met en contexte en ce moment.
De nos jour, il est possible de cr�er de films et des jeux vid�o qui donnent toujours un meilleur effet de r�alisme au fil des ann�es. Ce r�alisme provient de l'am�lioration constantes des techniques utilis�es lors de la cr�ation de ces m�dias. Un des aspects qui peut beaucoup influencer le r�alisme est la mod�lisation des objets dans la sc�ne. L'industrie tente toujours d'ajouter plus de d�tails aux mod�les 3D utilis�s dans le projet en ajoutant toujours plus de points et de triangles pour obtenir des formes qui imitent le mieux possible la forme r�elle. Souvent, on concentre les efforts surtout sur les d�tails dans les personnages principaux mais travailler sur le r�alisme de l'environnement peut aussi donner de bon, sinon meilleur r�sultat.

-- image de GTV IV ici + assassin's creed --

Puisqu'on am�liore le r�alisme de l'environnement autour des personnages, il faut ajouter plus d'information aux formes de la sc�ne. De plus, l'environnement doit toujours �tre plus grand pour donner un meilleur sentiment d'immersion. Il y a un probl�me en ce sens puisque plus il faut d'information pour sp�cifier des formes dans la sc�ne, plus il faut beaucoup de temps pour les artistes de cr�er cette information. Aussi, cr�er des environnement toujours plus grand demande aussi beaucoup de temps. Un moment donn�, ce n'est plus envisageable de demander � des gens de faire ce travail. Il faut donc trouver une fa�on d'utiliser la technologie de fa�on intelligente pour faire ce travail automatiquement ou semi-automatiquement. Si on regarde la figure (GTV IV), on voit qu'il y a beaucoup trop de d�tail dans l'environnement pour que quelqu'un ou une �quipe sp�cifie la position de tous les points de ce mod�le. Ce jeu se d�roule dans un environnement urbain c'est pourquoi l'�tude se poursuit pour cr�er des mod�les d'un environnement urbain. Pour obtenir un bon r�sultat, il est important d'avoir un bon sentiment d'immersion. Pour cela, il faut un bon r�alisme ainsi qu'une bonne vari�t� dans les b�timents de la ville. Par exemple, les images des jeux "Assassin's Creed" et "GTA IV" sont des bons exemples d'environnements r�ussis. La question ici est, comment utiliser la technologie pour arriver a de bons r�sultats de fa�on automatis�e?

%%%%%%%%%%%%%%%%%%%%%%%%%%%%%%%%%%%%%%%%%%%%%%%%%%%%%%%%%%%%
\section{Pr�sentation de l'�tat de l'art}
\label{sect:etatArt}
%%%%%%%%%%%%%%%%%%%%%%%%%%%%%%%%%%%%%%%%%%%%%%%%%%%%%%%%%%%%

\subsection{G�n�ration Proc�durale}
\label{ssect:procgen}
%%%%%%%%%%%%%%%%%%%%%%%%%%%%%%%%%%%%%%%%%%%%%%%%%%%%%%%%%%%%

-- image de mincraft ici --

La r�ponse � la question est la g�n�ration proc�durale. Cette technique a �t� beaucoup popularis� par un jeu vid�o indie nomm� Minecraft (voir figure minecraft). Par contre, il existe beaucoup de variantes � la g�n�ration proc�durale. L'article �tudi� se concentre sur la g�n�ration proc�durale de b�timents pour g�n�rer un environnement urbain r�aliste. Pour commencer, le fonctionnement de cette technique est d'utiliser des algorithme qui s'ex�cutent sur l'unit� de traitement d'un ordinateur pour g�n�rer un �l�ment. On ne sp�cifie pas directement le r�sultat de la g�n�ration ici parce qu'il est possible de g�n�rer pratiquement n'importe quel �l�ment gr�ce � cette technique. Elle offre aussi la possibilit� d'ajouter du hasard dans la g�n�ration pour ne pas obtenir toujours le m�me r�sutlat � chaque ex�cution. Elle est beaucoup utilis�e puisqu'il est possible d'ajouter de la vari�t� et de la rejouabilit� dans les jeux de nos jours. Par contre, au d�part, cette technique �tait beaucoup utilis�e pour �conomiser de l'espace lors de la cr�ation de jeux. Puisque certains �l�ments �taient g�n�r�s lorsque le jeu est en marche, il y avait moins de ressources � conserver en m�moire. Le premier exemple de g�n�ration proc�durale se trouve dans le jeu Akalabeth (1980) -- reference wiki + img -- o� les cartes �taient g�n�r�es. La g�n�ration proc�durale a beaucoup �t� popularis�e par l'arriv�e du jeu Minecraft. Certains jeux en font m�me un usage tr�s int�ressant. Par exemple, dans le jeu "Left 4 Dead", on g�n�re la difficult� de la mission courante d'apr�s les statistiques des joueurs ainsi que les �quipements qu'ils ont sur eux. On remarque donc que l'on peut g�n�rer n'importe quel �l�ment gr�ce � cette technique.

Avant l'�volution de la technique �labor�e dans l'article �tudi�, deux autres fa�on de g�n�rer des b�timents de fa�on proc�durale existaient.

\subsection{Premi�re Technique *auteur + ann�e*}
\label{ssect:first}
%%%%%%%%%%%%%%%%%%%%%%%%%%%%%%%%%%%%%%%%%%%%%%%%%%%%%%%%%%%%

Technique 1 par Parish and M�ller [2001]
Ces chercheurs ont montr� une m�thode pour g�n�rer un grand environnements urbains compos�s d'un grand nombre de b�timents. Chaque b�timent est constitu� de seulement de formes simples regroup�es en un modele. Ensuite, on ajoute des d�tails sur la fa�ade de chaque b�timent avec l'aide de shaders. Comme nous verrons plus tard dans l'�tude de la m�thode pr�sent�e dans l'article, cela peut mener � des probl�mes reli�s au r�alisme de la sc�ne. 
TODO: Trouver plus de d�tails sur cette technique

\subsection{Deuxi�me Technique *auteur + ann�e*}
\label{ssect:second}
%%%%%%%%%%%%%%%%%%%%%%%%%%%%%%%%%%%%%%%%%%%%%%%%%%%%%%%%%%%%

Technique 2 par Wonka et al. [2003]
Ici, les chercheurs ont voulu s'attaquer au probl�me de la g�n�ration de d�tails sur les fa�ade des b�timents g�n�r�s. Ils ont r�ussi � cr�er des d�tails qui sont en fait g�n�r�s � l'�tape g�om�trique. Ces d�tails sont donc compris dans le mod�le r�sultant et il n'est pas n�cessaire d'utiliser des nuanceurs pour ajouter du r�alisme aux fa�ades des b�timents. Par contre, avec cette technique, on s'attarde aux d�tails d'un seul b�timent � la fois. Il n'est donc pas possible de g�n�rer un grand environnement urbain comme dans la premi�re technique.

On remarque que l'on d�sire obtenir un m�lange de ces deux techniques. On d�sire un grand nombre de b�timents dans notre ville et que ces b�timents poss�dent des d�tails g�om�triques pour augmenter le r�alisme. Nous allons maintenant nous pencher sur la technique principale de l'article. On pr�cisera aussi en quoi cette m�thode est meilleure que les autres.
TODO: Trouver plus de d�tails sur cette technique

%%%%%%%%%%%%%%%%%%%%%%%%%%%%%%%%%%%%%%%%%%%%%%%%%%%%%%%%%%%%
\section{M�thode Choisie}
\label{sect:choisie}
%%%%%%%%%%%%%%%%%%%%%%%%%%%%%%%%%%%%%%%%%%%%%%%%%%%%%%%%%%%%

La technique utils�e ici se base sur les grammaires de formes. La premi�re partie de ce rapport se penche plus pr�cis�ment sur le fonctionnement de base d'une grammaire. Cette technique ,compar�e aux deux autres pr�sent�es au cours de ce rapport, g�n�re les d�tails des b�timents de fa�on g�om�trique par opposition aux nuanceurs. Il est donc possible de v�rifier s'il existe des intersection entre les d�tails g�n�r�s et les annuler puisque ces intersections enl�vent beaucoup de r�alisme � une sc�ne. Pour la deuxi�me technique pr�sent�e, l'avantage ici est que cette technique permet la cr�ation de grands environnements en gardant un temps d'ex�cution raisonnable. On pr�sente donc quelques �l�ments de base sur lesquels l'algorithme est bas� et plus tard, on montre les diff�rentes �tapes de l'algorithme plus en d�tail.

\subsection{Th�orie}
\label{ssect:theorie}
%%%%%%%%%%%%%%%%%%%%%%%%%%%%%%%%%%%%%%%%%%%%%%%%%%%%%%%%%%%%

\subsubsection{Une grammaire}
Dans le domaine informatique, une grammaire est d�finie par deux ensembles de symboles, un ensemble de r�gles et un symbole de d�part. Le but d'une grammaire est de passer du symbole de d�part et utiliser les r�gles pour passer � un �tat o� il n'y a que des symboles terminaux~\cite{CtxFreeGram}. Le premier ensemble de symbole sont les symboles terminaux. Ces symboles servent � marquer la fin d'une s�rie de changements sur un symbole. Un symbole terminal ne sera plus modifi� par les r�gles de la grammaire. Ensuite, il y a l'ensemble des symboles non-terminaux. Ces symboles repr�sentent des �tat transitifs qui seront modifi�s lors de des it�rations d'ex�cution de la grammaire. Pour finir, on d�finit les r�gles de la grammaires. Ces r�gles ont un format sp�cifique: non-terminal $\rightarrow$ serie de terminaux et non terminaux. Cette r�gle commence par annoncer sur quel symbole non-terminal il va agir et en quoi ce symbole sera transform�. On applique it�rativement une r�gle � la fois sur la chaine de symbole courante jusqu'� obtenir une cha�ne qui n'a que des symboles terminaux.

\subsubsection{Une grammaire de forme}
Dans le domaine de g�n�ration proc�durale de b�timent, on utilise la m�me id�e de base mais on adapte les principes au rendu 3D. Par exemple, les symboles terminaux et non-terminaux sont des formes de base~\cite{ShapeGram}. Aussi, les r�gles donnent les transformation g�om�triques qu'il faut faire et les formes qu'il faut ajouter pour obtenir une forme terminale pour terminer l'ex�cution de la grammaire de forme. Voici quelques exemples de r�gles dans une grammarie de forme~\cite{MainArticle}:
$$ A \rightarrow [ T(0,0,6) S(8,10,18) I("cube") ] T(6,0,0) S(7,13,18) I("cube") T(0,0,16) S(8,15,8) I("cylinder") $$
$$ fac \rightarrow Subdiv("Y",3.5,0.3,3,3,3) \{ floor | ledge | floor | floor | floor \} $$
On peut aussi sp�cifier des r�gles pour cr�er des sous-sections de formes g�ometriques. G�n�ralement, on donne une s�ries de longeurs (les diff�rentes s�parations) et sur quel axe (x, y ou z) on fait la s�paration. On peut utiliser ces s�paration pour cr�er des d�tails sur les fa�ades des b�timents qu'on g�n�re en ins�rant d'autres mod�les align�s sur ces s�parations.

\subsubsection{Les port�es}
Un autre �l�ment important de l'algorithme utilis� pour cette technique sont les port�es. Une port�e est une partie de l'espace 3D. Les transformations sp�cifi�es dans les r�gles de production sont affect�es sur les port�es~\cite{MainArticle}. Apr�s les transformations g�om�triques, si on ins�re un nouveau mod�le, il sera plac� dans la port�e et prendra la dimension et la position de la port�e. On sait donc que chaque forme dans une configuration de l'algorithme poss�de une port�e. De plus, il existe aussi une utilit� pour des port�es qui repr�sentent un espace 2D. Ces port�es peuvent �tre utilis�es pour g�n�rer un toit sur un b�timent de base. On d�fini une port�e en 2D align� sur un des murs du b�timent ou le dessus de la forme et on applique des transformations et on ins�re des nouvelles formes pour obtenir un toit sur notre structure de base.

\subsubsection{L'algorithme}
Pour commencer, une configuration est le nom qu'on donne � un nombre fini de formes dans l'espace 3D. Aussi, dans notre cas, un symbole est repr�sent� par une forme et ces symboles peuvent �tre soit non-terminaux ou terminaux. La d�finition plus pr�cise des symboles se trouve plus haut dans ce document. Au d�part l'algorithme doit recevoir une sp�cification d'un axiome qui est une configuration de formes simples pour d�marrer la g�n�ration. On proc�de ensuite par �tapes~\cite{MainArticle}: 

\begin{enumerate}
	\item On s�lectionne une forme (�tat) non-terminale qui est toujours active. � titre d'exemple, ce symbole est nomm� n
	\item On choisit ensuite une r�gle de production de la grammaire de forme � appliquer. Il faut que cette r�gle soit compos�e du symbole n comme partie gauche. Apr�s on g�n�re (� partir de la sp�cification de la r�gle choisie) un nouvel ensemble de formes. On nomme ce nouvel ensemble nNew
	\item On place la forme n comme inactive dans la liste des formes de la configuration courante. Pour finir on ajoute l'ensemble de formes nNew � la configuration et on recommence � l'�tapte no.1.
\end{enumerate}

\subsection{Exemples}
\label{ssect:ex}
%%%%%%%%%%%%%%%%%%%%%%%%%%%%%%%%%%%%%%%%%%%%%%%%%%%%%%%%%%%%

Les possibilit�s de g�n�ration sont pratiquement infinie. L'article �tudi� se penche plus en d�tails sur types d'exemples. Les trois premiers exemples montr�s dans l'article sont volontairement simples pour montrer la base de l'algorithme. Ils montrent 3 types de constructions: une maison simple, un bloc d'�difices � bureaux et une maison un peu plus pouss�e. Par apr�s, les exemples montr�s sont plus complexes et montrent une utilisation plus pouss�e de l'algorithme. Le premier exemple plus complexe est une extension du 3e exemple qui assemble plusieurs �l�ments pour cr�er un quartier de banlieue. Ensuite, le dernier exemple de l'article montre la g�n�ration d'une ville enti�re bas�e sur le style de Pompeii.

\subsubsection{Exemple 1}

\begin{figure}[h]
    \caption{Une maison simple~\cite{MainArticle}}
    \centering

    \includegraphics[width=7cm]{simpleEx.png}
    \label{simpEx}
\end{figure}

Cet exemple montre la fa�on de cr�er la structure la plus simple avec cet algorithme pour le corps principal de la maison et la forme la plus simple de toits. Les murs sont simplement construit � partir d'une forme deux dimensions avec laquelle une extrusion a �t� faite pour obtenir un prisme en trois dimensions. Pour ce qui est du toit, il est construit � partir deux 2 plans en 2D qui se rencontrent au milieu du prisme de la structure du b�timent.

\subsubsection{Exemple 2}

\begin{figure}[h]
    \caption{Un bloc d'�difices a bureaux~\cite{MainArticle}}
    \centering

    \includegraphics[width=7cm]{blocEx.png}
    \label{blocEx}
\end{figure}

Cet exemple tente de montrer qu'il est possible de g�n�rer plus qu'un seul b�timent � la fois � l'aide de cet algorithme. L'algorithme utilise encore des formes 2D comme une base et l'extrusion de ces formes donne la forme g�n�rale des b�timents. La distribution des formes 2D au d�part d�finissent la densit� de la ville que l'on veut obtenir. Par exemple, dans la figure~\ref{blocEx} il y a une grande densit� de grand b�timents. Aucun d�tails n'a �t� g�n�rer sur les fa�ades des b�timents dans cet exemple puisque les d�tails seront montr�s dans les exemples suivants.

\subsubsection{Exemple 3}

\begin{figure}[h]
    \caption{Une maison de banlieue plus �volu�e~\cite{MainArticle}}
    \centering

    \includegraphics[width=7cm]{banEx.png}
    \includegraphics[width=7cm]{groupEx.png}
    \label{groupEx}
\end{figure}

Dans cet exemple, on ajoute des d�tails autour de la maison. Par exemple, on ajoute une cour, un stationnement, et de la v�g�tation. Ces d�tails sont g�n�r�s de la m�me fa�on que les �l�ments de base. On ajoute aussi des trottoirs autour de la propri�t�.

On peut aussi utiliser plusieurs maisons de ce type et les rassembler pour cr�er un quartier de banlieue (figure~\ref{groupEx}). Ce rassemblement ne veut pas seulement dire qu'on execute plusieurs fois l'algorithme et qu'on combine les mod�les r�sultants en un seul. On peut tout simplement modifier les r�gles de l'algorithme pour faire en sorte qu'il soit possible de g�n�rer plus qu'un seul b�timent, mais conserver les r�gles qui ajoutent des d�tails de v�g�tation et des trottoirs. De la m�me fa�on, on peut g�n�rer une ville plus grande en utilisant les r�gles de l'exemple 2 comme une base.

% \subsection{Outil Existant}
% \label{ssect:outil}
%%%%%%%%%%%%%%%%%%%%%%%%%%%%%%%%%%%%%%%%%%%%%%%%%%%%%%%%%%%%

% Il existe un outils qui fait usage de la technique d�crite dans l'article. Cet outils s'apelle CityEngine. On remarque beaucoup des principes pr�sent� dans l'article dans le vid�o d'introduction de l'outil[reference].

%%%%%%%%%%%%%%%%%%%%%%%%%%%%%%%%%%%%%%%%%%%%%%%%%%%%%%%%%%%%
\section{Conclusion}
\label{sect:conclusion}
%%%%%%%%%%%%%%%%%%%%%%%%%%%%%%%%%%%%%%%%%%%%%%%%%%%%%%%%%%%%

Voil� une courte explication du fonctionnement de l'algorithme pr�sent� dans l'article �tudi�. L'algorithme offre des possibilit�s de g�n�ration infinie. La puissance de l'algorithme provient de la d�finition des r�gles. Comme prouv� dans l'exemple de la ville dans le style de Pompeii~\cite{MainArticle}. Cette technique de g�n�ration a �t� d�velopp�e pour combiner les avantages de deux autres techniques qui ont �t� mises au point auparavant. De plus, il est facilement possible d'ajouter des �l�ments de hasard dans la g�n�ration. De cette fa�on, on peut cr�er plus d'une ville avec la m�me s�rie de r�gles.

Il reste quelques aspects de l'algorithme qui n'ont pas �t� couvert dans l'article �tudi� au cours de la session. Par exemple, la performance possible de l'algorithme ainsi que la parall�lisation de ce dernier. Il aurait �t� int�ressant de voir quelles parties de l'algorithme sont parall�lisable et quel facteur d'am�lioration il est possible d'obtenir avec la version parall�le de l'algorithme.

Pour finir, il existe un outil qui utilise cet algorithme pour faire une g�n�ration de mod�les de villes automatiquement~\cite{EsriCityEngine}. Dans le vid�o d'introduction de cet outil, on remarque la plupart des sujets qui on �t� abord�s dans ce rapport. On remarque aussi que la performance de l'algorithme est assez bonne puisque les r�sutlats de l'algorithme sont rendus en temps r�el.


\eject

\addcontentsline{toc}{section}{\refname}
\bibliography{biblio}

\end{document}
