\documentclass[12pt]{article}
    \usepackage{times}
    \usepackage[latin1]{inputenc}
    \usepackage[frenchb]{babel}
    \usepackage{graphicx}
    \usepackage[numbers]{natbib}
    \usepackage{amsfonts}
    \usepackage{amsmath}

    \numberwithin{equation}{section}
    \numberwithin{figure}{section}
    \numberwithin{table}{section}

    \usepackage[nooneline]{subfigure}
    \usepackage{bm}
%    \usepackage{prelim2e}
    \usepackage{booktabs}
    \usepackage{color}
    \usepackage{rotating}
    \usepackage{afterpage}
%    \usepackage[draft]{fixme}
    \usepackage{refstyle}
        \newref{equ}{refcmd=(\ref{#1}),name=�quation~,Name=�quation~,names=�quations~,Names=�quation~,lsttxt={\ et\ },rngtxt={\ �\ }}
        \newref{fig}{name=figure~,names=figures~,Name=Figure~,Names=Figures~,lsttxt={\ et\ },rngtxt={\ �\ }}
        \newref{subfig}{name=figure~,names=figures~,Name=Figure~,Names=Figures~,lsttxt={\ et\ },rngtxt={\ �\ }}
        \newref{chap}{name=chapitre~,names=chapitres~,Name=Chapitre~,Names=Chapitres~,lsttxt={\ et\ },rngtxt={\ �\ }}
        \newref{sect}{name=section~,names=sections~,Name=Section~,Names=Sections~,lsttxt={\ et\ },rngtxt={\ �\ }}
        \newref{ssect}{name=section~,names=sections~,Name=Section~,Names=Sections~,lsttxt={\ et\ },rngtxt={\ �\ }}
        \newref{sssect}{name=section~,names=sections~,Name=Section~,Names=Sections~,lsttxt={\ et\ },rngtxt={\ �\ }}
        \newref{tab}{name=tableau~,names=tableaux~,Name=Tableau~,Names=Tableaux~,lsttxt={\ et\ },rngtxt={\ �\ }}
        \newref{step}{name=�tape~,names=�tapes~,Name=�tape~,Names=�tapes~,lsttxt={\ et\ },rngtxt={\ �\ }}
        \newref{annex}{name=annexe~,names=annexes~,Name=Annexe~,Names=Annexes~,lsttxt={\ et\ },rngtxt={\ �\ }}
    \usepackage{fancyhdr}

    \setlength{\textwidth}{175mm}
    \setlength{\textheight}{210mm}
    \setlength{\headsep}{2mm}
    \setlength{\topmargin}{-3mm}
    \setlength{\unitlength}{1mm}
    \setlength{\itemsep}{0mm}
    \setlength{\columnsep}{0.3125in}
    \setlength{\headheight}{10mm}
    \setlength{\parindent}{1pc}
    \setlength{\oddsidemargin}{-.304in}
    \setlength{\evensidemargin}{-.304in}
    \setlength{\abovedisplayskip}{-2mm}
    \setlength{\belowdisplayskip}{-2mm}
    \setlength{\abovedisplayshortskip}{-2mm}
    \setlength{\belowdisplayshortskip}{-2mm}
    \setlength{\topsep}{0mm}
    \addtolength{\headsep}{10mm}

%    \renewcommand{\subfigcapskip}{3pt}
%    \renewcommand{\subfigtopskip}{2pt}
%    \renewcommand{\subfigbottomskip}{3pt}
%    \setlength{\textfloatsep}{6pt plus 3pt minus 3pt}
%    \setlength{\floatsep}{6pt plus 3pt minus 3pt}
%    \setlength{\abovecaptionskip}{5pt}


    % Allows to have list item with less space between each element of the list
    % Use \noitemsep or \doitemsep before
    \let\origEnumerate =\enumerate
    \let\origItemize =\itemize
    \let\origDescription =\description
    \def\Nospacing{\itemsep=0pt\topsep=3pt\partopsep=0pt%
    \parskip=0pt\parsep=0pt}
    \def\noitemsep{% Redefine the environments in terms of the original values
        \renewenvironment{itemize}{\origItemize\Nospacing}{\endlist}
        \renewenvironment{enumerate}{\origEnumerate\Nospacing}{\endlist}
        \renewenvironment{description}{\origDescription\Nospacing}{\endlist}
    }
    \def\doitemsep{% Redefine the environments to the original values
        \renewenvironment{itemize}{\origItemize}{\endlist}
        \renewenvironment{enumerate}{\origEnumerate}{\endlist}
        \renewenvironment{description}{\origDescription}{\endlist}
    }

    % Allows to ignore continued figure in the TOC
    \makeatletter
    \def\afterfi#1\fi{\fi#1}
    \let\ORIGaddtocontents\addtocontents
    \newcommand*\dontaddtolof[2]{\edef\temp{#1}%
    \ifx\temp\ext@figure\else\afterfi\ORIGaddtocontents{#1}{#2}\fi}
    \makeatletter
    \newcommand*\ignorelof{\let\addtocontents\dontaddtolof}
    \newcommand*\obeylof{\let\addtocontents\ORIGaddtocontents}

    \bibliographystyle{plainnat-fr}

%    \linespread{1.6}

    \pagestyle{fancy}

    \lhead{\small{DI, Universit� de Sherbrooke}}
    \rhead{\small{g�n�ration proc�durale de b�timents}}

\begin{document}

\title{G�n�ration proc�durale de b�timents}

\author{Alexandre Brochu}
\date{24 / 04 / 2015}

\maketitle

\thispagestyle{empty}

\vspace*{-5mm}

\begin{center}
\small{
    D�partement d'informatique \\
    Universit� de Sherbrooke \\
    Sherbrooke (Qc), Canada, J1K 2R1 \\
    Alexandre.Brochu@usherbrooke.ca
}
\end{center}

\vspace*{5mm}

\begin{center}
\textbf{- Rapport de recherche no 1 -}
\end{center}

\vspace*{5mm}

\begin{abstract}
    le travail reli� � la cr�ation de mod�les 3D de villes pour cr�er un r�alisme dans un environnement urbain dans les films ainsi que dans les jeux vid�o devient de plus en plus complexe. On recherche des fa�ons d'automatiser ce travail � l'aide d'algorithmes qui se basent sur la g�n�ration proc�durale. Ici, on parle d'un de ces algorithmes un peu plus en d�tail.
\end{abstract}

%-------------------------------------------------------------------------

\renewcommand{\listtablename}{Liste des tableaux}
\renewcommand{\refname}{Bibliographie}
\renewcommand{\listfigurename}{Liste des figures}

\eject

    \tableofcontents

\eject

    \addcontentsline{toc}{section}{\listtablename}
    \listoftables
    \addcontentsline{toc}{section}{\listfigurename}
    \listoffigures

\eject

%-------------------------------------------------------------------------

%%%%%%%%%%%%%%%%%%%%%%%%%%%%%%%%%%%%%%%%%%%%%%%%%%%%%%%%%%%%
\section{Mise en contexte}
\label{sect:miseEnCtx}
%%%%%%%%%%%%%%%%%%%%%%%%%%%%%%%%%%%%%%%%%%%%%%%%%%%%%%%%%%%%

%You can use~\citep{deriche95, tschumperle02, weickert97} or~\citet{deriche95,
%tschumperle02, weickert97} or~\citeauthor{deriche95, tschumperle02, weickert97}
%or~\citeyear{deriche95, tschumperle02, weickert97}.

%Example of equation
%\begin{equation}\label{equ:HeatConserv}
%    \nabla \cdot (\kappa \nabla u) = f.
%\end{equation}

%Reference to \equref{HeatConserv}.

%Reference to \sectref{one}.
%%%%%%%%%%%%%%%%%%%%%%%%%%%%%%%%%%%%%%%%%%%%%%%%%%%%%%%
On met en contexte en ce moment.

%%%%%%%%%%%%%%%%%%%%%%%%%%%%%%%%%%%%%%%%%%%%%%%%%%%%%%%%%%%%
\section{Probl�matique}
\label{sect:problem}
%%%%%%%%%%%%%%%%%%%%%%%%%%%%%%%%%%%%%%%%%%%%%%%%%%%%%%%%%%%%

Intro to \sectref{problem}.

Ici, on explique la probl�matique que l'algorithme pr�sent� dans l'article tente de solutionner.
%%%%%%%%%%%%%%%%%%%%%%%%%%%%%%%%%%%%%%%%%%%%%%%%%%%%%%%%%%%%
\section{Pr�sentation de l'�tat de l'art}
\label{sect:etatArt}
%%%%%%%%%%%%%%%%%%%%%%%%%%%%%%%%%%%%%%%%%%%%%%%%%%%%%%%%%%%%

Intro to \sectref{etatArt}.


\subsection{G�n�ration Proc�durale}
\label{ssect:procgen}
%%%%%%%%%%%%%%%%%%%%%%%%%%%%%%%%%%%%%%%%%%%%%%%%%%%%%%%%%%%%

G�N�RATION PROC�DURALE

\subsection{Premi�re Technique *auteur + ann�e*}
\label{ssect:first}
%%%%%%%%%%%%%%%%%%%%%%%%%%%%%%%%%%%%%%%%%%%%%%%%%%%%%%%%%%%%

PREMI�RE TECHNIQUE EXPLIQU�E

\subsection{Deuxi�me Technique *auteur + ann�e*}
\label{ssect:second}
%%%%%%%%%%%%%%%%%%%%%%%%%%%%%%%%%%%%%%%%%%%%%%%%%%%%%%%%%%%%

DEUXI�ME TECHNIQUE EXPLIQU�E
%%%%%%%%%%%%%%%%%%%%%%%%%%%%%%%%%%%%%%%%%%%%%%%%%%%%%%%%%%%%
\section{M�thode Choisie}
\label{sect:choisie}
%%%%%%%%%%%%%%%%%%%%%%%%%%%%%%%%%%%%%%%%%%%%%%%%%%%%%%%%%%%%

Intro to \sectref{choisie}.


\subsection{Th�orie}
\label{ssect:theorie}
%%%%%%%%%%%%%%%%%%%%%%%%%%%%%%%%%%%%%%%%%%%%%%%%%%%%%%%%%%%%

TH�ORIQUEMENT

\subsection{Exemples}
\label{ssect:ex}
%%%%%%%%%%%%%%%%%%%%%%%%%%%%%%%%%%%%%%%%%%%%%%%%%%%%%%%%%%%%

EXEMPLES

\subsection{Outil Existant}
\label{ssect:outil}
%%%%%%%%%%%%%%%%%%%%%%%%%%%%%%%%%%%%%%%%%%%%%%%%%%%%%%%%%%%%

OUTIL EXISTANT
%%%%%%%%%%%%%%%%%%%%%%%%%%%%%%%%%%%%%%%%%%%%%%%%%%%%%%%%%%%%
\section{Conclusion}
\label{sect:conclusion}
%%%%%%%%%%%%%%%%%%%%%%%%%%%%%%%%%%%%%%%%%%%%%%%%%%%%%%%%%%%%

This is the end.


\eject

\addcontentsline{toc}{section}{\refname}
\bibliography{biblio}

\end{document}
