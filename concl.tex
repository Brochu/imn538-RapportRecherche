%%%%%%%%%%%%%%%%%%%%%%%%%%%%%%%%%%%%%%%%%%%%%%%%%%%%%%%%%%%%
\section{Conclusion}
\label{sect:conclusion}
%%%%%%%%%%%%%%%%%%%%%%%%%%%%%%%%%%%%%%%%%%%%%%%%%%%%%%%%%%%%

Voil� une courte explication du fonctionnement de l'algorithme pr�sent� dans l'article �tudi�. L'algorithme offre des possibilit�s de g�n�ration infinie. La puissance de l'algorithme provient de la d�finition des r�gles. Comme prouv� dans l'exemple de la ville dans le style de Pompeii~\cite{MainArticle}. Cette technique de g�n�ration a �t� d�velopp�e pour combiner les avantages de deux autres techniques qui ont �t� mises au point auparavant. De plus, il est facilement possible d'ajouter des �l�ments de hasard dans la g�n�ration. De cette fa�on, on peut cr�er plus d'une ville avec la m�me s�rie de r�gles.

Pour finir, il existe un outil qui utilise cet algorithme pour faire une g�n�ration de mod�les de villes automatiquement~\cite{EsriCityEngine}.
